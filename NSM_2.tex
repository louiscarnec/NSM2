\documentclass[a4paper,11pt]{article}
\usepackage{graphicx}
\usepackage{lscape}
\usepackage{capt-of}
\usepackage{fancyhdr}
\usepackage{caption}
\usepackage{subcaption}
\usepackage{tikz}
\usetikzlibrary{arrows.meta}
\usepackage{enumerate}
\usepackage{siunitx}
\usepackage{eurosym}
\usetikzlibrary{quotes,angles,positioning}
\usepackage{standalone}
\usepackage{multirow}
\usepackage{geometry} % to change the page dimensions
\geometry{a4paper} 
\pagestyle{fancy}
\lhead{}
\rhead{Louis Carnec \\ 15204934}
\renewcommand{\headrulewidth}{0pt}
\setlength\parindent{0pt}
\usepackage{amstext} % for \text
\DeclareRobustCommand{\officialeuro}{%
  \ifmmode\expandafter\text\fi
  {\fontencoding{U}\fontfamily{eurosym}\selectfont e}}
\begin{document}



\begin{titlepage}

\newcommand{\HRule}{\rule{\linewidth}{0.5mm}} % Defines a new command for the horizontal lines, change thickness here

\center % Center everything on the page
 

%----------------------------------------------------------------------------------------
%	TITLE SECTION
%----------------------------------------------------------------------------------------

\HRule \\[0.4cm]
{ \huge \bfseries Network Software Modelling Assignment 2}\\[0.4cm] % Title of your document
\HRule \\[1.5cm]
 
%----------------------------------------------------------------------------------------
%	AUTHOR SECTION
%----------------------------------------------------------------------------------------

\begin{minipage}{0.4\textwidth}
\begin{flushleft} \large
\emph{Students:}\\
Louis \textsc{Carnec} \\% Your name
Vijay \textsc{Katta}\\
Adedayo \textsc{Adelowokan}  
\end{flushleft}
\end{minipage}
~
\begin{minipage}{0.4\textwidth}
\begin{flushright} \large
\emph{Student \#:} \\
15204934 \\ 15202724 \\15204151 % Supervisor's Name
\end{flushright}
\end{minipage}\\[4cm]

% If you don't want a supervisor, uncomment the two lines below and remove the section above
%\Large \emph{Author:}\\
%John \textsc{Smith}\\[3cm] % Your name

%----------------------------------------------------------------------------------------
%	DATE SECTION
%----------------------------------------------------------------------------------------

{\large \today}\\[3cm] % Date, change the \today to a set date if you want to be precise

%----------------------------------------------------------------------------------------
%	LOGO SECTION
%----------------------------------------------------------------------------------------

%\includegraphics{Logo}\\[1cm] % Include a department/university logo - this will require the graphicx package
 
%----------------------------------------------------------------------------------------

\vfill % Fill the rest of the page with whitespace

\end{titlepage}

Questions:

How quickly should a given country stop letting flights in to avoid the disease reaching it? when is too late? Different results for iceland and uk for example?...
Centrality of country: how quick does it get disease

simulation property: screening for disease - catch given percentage -  effect on spread

simulation property - chance of receiving - centrality
probability of country with disease of spreading to other countries weighted on population of that country
probability increases at each timestep to account for fact that disease spreading within the country




\section{Read world phenomenon : Spread of disease through airport networks}

In this simulation we will model the spread of disease through a network of airports. This topic is 

This simulation is inspired by the work of Nichola Yager which models the spread of pathogens through airports using a directed graph. We extend his work by simulating disease spread 

\section{Describe briefly your choice of graph model and how it corresponds to the real-world phenomenon, e.g. your choice of directed versus undirected edges, edge weights, allowing or disallowing self-loops, etc.}

Edge weights - average of two nodes(airports)

Multiple routes between any two nodes are removed.


Airports with no flights are removed.

Only airports within USA are used.

\section{With the aid of a diagram, describe the simulation rules, e.g. how agents change state and what causes them to send messages.}

Write about SIR model - Susceptible, Infected, Recovered and Dead.


How the first node/nodes are infected - initialised with probability


Time Steps


 Node Colour - green : Susceptible, yellow: infected, white: recovered, red: dead

Edges with higher weight have higher probability of spreading the disease.

Recovered from Infection are not susceptible again

\section{Program your simulation in Python.}

\section{State one or more simulation properties which you wish to investigate and which graph properties you hypothesize they may depend on.}

radius


diameter


centrality


order

\section{Carry out experiments to measure these properties in different types and sizes of graph (at least one real-world graph, and at least one scalable graph model such as the Erdos-Renyi random graph model, at multiple sizes). Present a table of data showing your results.}


Graph property tests - Tables of results

Results Diagrams/figures from Python code


Scalability  test results along with tables and figures


\section{State your conclusions, based on your data.}

Discuss the results


Statistics

\clearpage
\bibliographystyle{acm}
\bibliography{NSM}

\end{document}